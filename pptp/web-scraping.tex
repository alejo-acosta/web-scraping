%-------------------------------------------------------
\documentclass{beamer}
\usepackage{lmodern}
\usefonttheme[onlymath]{serif}
% \usetheme{beaver}
% \usetheme{Hannover}
% \usetheme{Singapore}
% \usetheme{CambridgeUS}
% \usetheme{Boadilla}
\usetheme{Madrid}
% \usecolortheme{beaver}
\colorlet{beamer@blendedblue}{green!35!black}
% \usetheme{Frankfurt}
% \usecolortheme{beaver}
% \setbeamertemplate{navigation symbols}{}

%\hypersetup{pdfpagemode=FullScreen}

\usepackage[mode=build]{standalone}
\usepackage[absolute,overlay]{textpos}
\usepackage[utf8]{inputenc}
\usepackage[spanish]{babel}
\usepackage{hyperref}
\usepackage{tikz}
\usepackage{graphicx}
\usepackage{booktabs}
\usepackage{caption}
\usepackage[rightcaption]{sidecap}
\usepackage{caption}
\usepackage{makecell}
% \usepackage{emoji}
% \graphicspath{{../03-outputs/figures/}}

\usepackage{apacite}
%\usepackage{natbib}
\usepackage{etoolbox}
\renewenvironment{APACrefURL}[1][]{}{}
\AtBeginEnvironment{APACrefURL}{\renewcommand{\url}[1]{}}
\renewcommand{\doiprefix}{doi:~\kern-1pt}

\bibliographystyle{apacite}

\title{Web scraping}
\subtitle{Para economistas}
\author{Alejandro Acosta León}

\date{\today}

\logo{
\includegraphics[width=2.5cm]{C:/Users/alejo/OneDrive - Universidad de Las Américas/Ico/economia-logo.png}
}


%-------------------------------------------------------
\begin{document}
\maketitle


\begin{frame}
	\frametitle{Material}
	\begin{alertblock}{ }
		\centering
		\textbf{El material de este taller está disponible en:} \\ \url{https://github.com/alejo-acosta/web-scraping}
	\end{alertblock}
	\centering
	Pueden descargar el material en la pestaña de código y luego en "Download ZIP".  \\
	O pueden clonar el repositorio con el siguiente comando: \\
	\texttt{git clone https://github.com/alejo-acosta/web-scraping }

\end{frame}

\begin{frame}
	\frametitle{Expectativas}
	\begin{alertblock}{ }
		\centering
		\textbf{¿Qué esperas aprender en este taller?}
	\end{alertblock}
	\centering
	Escríbanlo en el siguiente \href{https://miro.com/app/board/uXjVKA1HUxk=/?share_link_id=961163594502}{Link}

\end{frame}


\begin{frame}
	\frametitle{Agenda - Teoría}
	\underline{\textbf{Conceptos}}
	\begin{itemize}
		\item ¿Qué es web scraping?
		\item Problemas éticos, legales y mal uso del web scraping.
		\item Alternativas al web scraping.
		\item Herramientas y librerías necesarias.
	\end{itemize}
	
	\underline{\textbf{Entorno}}
	\begin{itemize}
		\item Instalación de Python y librerías necesarias (requests, BeautifulSoup, pandas, selenium).
		\item Configuración del entorno de desarrollo (puede ser Jupyter Notebook, VSCode, etc.).
	\end{itemize}	
	
\end{frame}


\begin{frame}
	\frametitle{Agenda - Práctica}
	\underline{\textbf{Nivel 1: data estructurada en HTML}}
	\begin{itemize}
		\item Pandas :)
	\end{itemize}
	
	\underline{\textbf{Nivel 2: data semi-estructurada}}
	\begin{itemize}
		\item Inspección de elementos HTML.
		\item Extracción de datos con BeautifulSoup.
		\item Descarga de archivos.
	\end{itemize}

	\underline{\textbf{Nivel 3: data no estructurada y que no usa HTML}}
	\begin{itemize}
		\item Automatización web con Selenium.
		\item Extracción de información relevante.
	\end{itemize}
\end{frame}

\begin{frame}
	\frametitle{¿Qué es web scraping?}
	\begin{alertblock}{ }
		\centering
		Web scraping son técnicas que consisten en extraer información de sitios web. Estas técnicas pueden ser manuales o automatizadas.
	\end{alertblock}
	
	
	\begin{itemize}
		\item El problema es que los sitios web no están diseñados para ser leídos por máquinas, sino por humanos. Por lo tanto, debemos 'enseñar' a nuestra computadora a leer estos sitios web.  \\
		\item Además, el internet puede ser muy ordenado o completamente caótico. Por lo tanto, debemos ser creativos y flexibles en la forma en que extraemos la información.
		\item Normalmente el web scraping se puede resumir en 3 'proceso' \\
		\begin{enumerate}
			\item Parsing o análisis del código.
			\item Automatización (bots).
			\item Extracción de la información.
		\end{enumerate}
	\end{itemize}
\end{frame}

\begin{frame}
	\begin{alertblock}{ }
		\centering
		Imaginémonos que vivimos dentro del internet, en la gran nación democrática y soberana de Weblandia.
	\end{alertblock}
	\centering
	Así como en la vida real, Weblandia tiene bienes públicos y privados. \\
	Normalmente, el web scraping se hace sobre 'bienes públicos'.  \\
	\includegraphics[width=.2\linewidth]{wikipedia.jpg} \large{vs} 
	\includegraphics[width=.2\linewidth]{netflix.png}  \\

	Los bienes públicos son no excluyentes y no rivales. \\
	¿Pero realmente es así en Weblandia o incluso en la vida real?

\end{frame}

\begin{frame}
	\frametitle{Problemas éticos y legales}
	\begin{alertblock}{ }
		\centering
		\textbf{¿Cuáles?}
	\end{alertblock}
	
\end{frame}


\begin{frame}
	\frametitle{Ejemplos de uso del web scraping en la literatura}
	\begin{itemize}
		\item Cavallo, Alberto, and Roberto Rigobon. 2016. "The Billion Prices Project: Using Online Prices for Measurement and Research." \textit{Journal of Economic Perspectives, 30 (2): 151-78.}
		\item Glaeser, Edward L., Hyunjin Kim, and Michael Luca. 2018. "Nowcasting Gentrification: Using Yelp Data to Quantify Neighborhood Change." \textit{AEA Papers and Proceedings, 108: 77-82.}
	\end{itemize}
\end{frame}

\begin{frame}
	\frametitle{Cuando no usar web scraping}
	Hacer web scraping es lento, tedioso y muchas veces poco eficiente.
	\begin{itemize}
		\item Si la información está disponible en un formato estructurado (CSV, Excel, SQL, etc.).
		\item Si la información está disponible a través de una API.
		\item Si la información es sensible o privada.
		\item Con fines maliciosos (spam, phishing, DDoS, brute force hacking).
		\item Si nuestro proceso acapara todos los recursos del servidor.
		\item Si se hace por 'moda'.
	\end{itemize}
	\begin{block}{Revisar:}
		Foerderer, J. (2023). Should we trust web-scraped data?. \textit{arXiv preprint arXiv:2308.02231.}
	\end{block}
\end{frame}

\begin{frame}
	\begin{block}{}
		\centering
		\large{Manos a la obra, preparemos nuestro entorno de trabajo.}
	\end{block}
\end{frame}

\end{document}